\begin{titulo}
    \raggedright\raggedright\textbf{4. CONCLUSÃO}
\end{titulo}
\par
Em 1881, Joaquim Maria Machado de Assis, popularmente conhecido como Machado de Assis, laça sua obra prima \textit{Memórias Póstumas de Brás Cubas} \cite{de1998memorias}, a qual Machado termina com a seguinte frase: "Não tive filhos, não transmiti a nenhuma criatura o legado de nossa miséria.". Em seu livro, essa frase captura a perspectiva do personagem Brás Cubas, que narra sua vida após a morte. Brás Cubas sugere que a vida é marcada por decepções, sofrimento e desilusões, e ele se considera isento de responsabilidades e culpas por não ter trazido mais seres ao mundo para enfrentar essas dificuldades. Essa frase pode ser interpretada como uma crítica à condição humana, destacando a fugacidade da existência e a inevitabilidade do sofrimento. Ela também revela a visão irônica de Machado de Assis em relação à vida e à sociedade, questionando valores e expectativas tradicionais.
\par
Diante do exposto, quando falamos do mundo real, a vida que presentemos a cada dia nesse planeta, não podemos ter uma visão pessimista tampouco desilusória como de Brás Cubas. Faz-se fulcral, pois assim é civilizatório, que lutemos por nossos direitos, e sobretudo lutemos pela disrupção das desigualdades sociais as quais o capitalismo nos impõe. Pois assim, legar-se-á às futuras gerações não a nossa miséria, mas a luta de nossas histórias, aquilo que nos marcam como humanos, o reconhecimento do outro também como humano.