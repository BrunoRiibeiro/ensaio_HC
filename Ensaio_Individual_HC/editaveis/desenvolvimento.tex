\begin{titulo}
    \raggedright\raggedright\textbf{2. A INTELIGENCIA ARTIFICIAL NA SOCIEDADE DO CANSAÇO}
\end{titulo}
\par
A inteligência artificial (IA) tem passado por avanços notáveis e promissores no campo da tecnologia nas últimas décadas. Ela diz respeito à habilidade das máquinas e sistemas computacionais de imitar a inteligência humana, adquirindo conhecimento, raciocinando, identificando padrões e tomando decisões de maneira independente.
\par
Foi em 30 de novembro de 2022 que a empresa OpenAI lançou seu mais novo, e provavelmente  mais famoso, serviço, o ChatGPT  (Generative Pre-trained Transformer) é um sistema de inteligência artificial desenvolvido pela OpenAI, o qual a partir de um comando de texto é capaz de responder e produzir praticamente qualquer tipo de resposta imaginável. Desde então, o mundo mudou, as conversas não são mais as mesmas, o ChatGPT foi um marco na história não só da tecnologia mas da sociedade como um todo. Atualmente, no mês e ano que este ensaio está há ser escrito, o maior tópico de discussão em todas as esferas da sociedade é a evolução da inteligencia artificial, esse tema vem aparecendo desde vídeos e blogs na internet até reportagens jornalisticas e artigos acadêmicos. Assim, não é de se surpreender que chega-se também ao debate público, a superação e não maior necessidade do ser humano no processo produtivo e no mercado de trabalho.
\par
A superação da mão de obra humana no processo produtivo não é um assunto de todo novo, na verdade essas ideias vêm se propagando, com muita força, desde as primeiras revoluções industriais na Inglaterra, a diferença é que hoje o trabalho a ser substituído não é mais apenas o trabalho árduo e braçal das fábricas inglesas, mas o trabalho humano intelectualizado. Consoante a isso, é necessário avaliar e analisar quais os reais impactos que essas substituições podem causar na sociedade capitalista como um todo.
\par
A sociedade contemporânea, descrita por Byung-Chul Han, filósofo sul-coreano, como sociedade do cansaço \cite{han2015sociedade}, vem sendo corriqueiramente bombardeada por ideologias neoliberais, as quais provocam o ser humano a fadigar-se em uma constante busca de sucesso, por meio de um alto desempenho, enquanto, afoga-se no modus operandi de produção capitalista burguês. Han argumenta que, diferentemente da sociedade disciplinar descrita por Michel Foucault, onde o poder era exercido por meio da repressão e vigilância, a sociedade atual opera por meio de um incentivo constante à produtividade e eficiência. Nesse contexto, os indivíduos são impulsionados a se tornarem empreendedores de si mesmos, buscando incessantemente o sucesso e a autorrealização.
\par
A partir de uma busca incessante pelo sucesso, essa sociedade, em uma busca ilusória de indivíduos empreendedores de si mesmos, vem se afastando do real significado social do trabalho, transformando-o em uma incessante corrida na esteira do capitalismo. No \textit{Manifesto Comunista}, Marx e Engels \cite{ManifestoComunista} argumentam que o capitalismo é uma forma de organização social caracterizada pela exploração e alienação do proletariado. Através da análise materialista dialética, eles sustentam que o capitalismo contém em si as sementes de sua própria destruição. A luta de classes, a crescente concentração de riqueza nas mãos da burguesia e a exploração da força de trabalho são contradições fundamentais que levam à necessidade de uma revolução proletária.
\par
O acumulo e concentração de riquezas nas mãos da burguesia vem-se - no contexto marxista, pela mais-valia -, desse modo, refere-se à diferença entre o valor produzido pelo trabalho de um proletário e o valor que ele recebe como salário. Marx argumenta que, incerne ao capitalismo, os trabalhadores produzem valor excedente para além do valor que é pago a eles, e esse valor é apropriado pelos proprietários dos meios de produção, os capitalistas. Essa exploração é considerada uma fonte central da acumulação de capital e da desigualdade na sociedade capitalista.
\par
Com os recentes avanços relacionados à inteligencia artificial, é notável que o trabalho a ser substituído não é mais o trabalho pesado e degradante ao homem mas sim o trabalho intelectual, hoje não é mais um diploma que garantirá seu emprego, pois no caminho que seguimos essas tecnologias galgam transpor a mão de obra humana qualificada. Assim, nota-se crescentes discussões sobre a qualidade dessas inteligencias ao redigir textos, produzir desenhos ou outras obras de arte, codificar programas, fazer cálculos entre outras tarefas, de alto nível de qualificação técnica. Porém, tais discussões, muitas vezes, são em vão, haja vista que o que importa nesse processo não é necessariamente chegar a perfeição destas tarefas, e sim, chegar a um ponto o qual a redução de custos ao capitalista para com seus empregados seja benéfica o suficiente, de modo que a extração de masi-valia seja a máxima possível.
\par
Quando pensamos no avanço tecnológico de inteligencias artificiais, na sociedade do cansaço de Han como uma forma de atenuar a desigualdade social, percebe-se não a diminuição do trabalho da classe trabalhadora mas sim seu uso como uma ferramenta para a manutenção do status quo. Logo, tendo em vista que o trabalho a ser superado é o trabalho qualificado, não será a classe proletária que terá uma melhoria em sua qualidade de vida, com menos horas de trabalho, mais tempo de lazer e menos esforço físico, muito pelo contrário. Nesse âmbito, haverá uma incrível disruptura na forma como entendemos as classes sociais hoje, com a superação da mão de obra humana qualificada, teremos essa parcela da sociedade sendo removida de seus postos e designadas ao trabalho braçal novamente, voltaremos a um mundo o qual ou se é incrivelmente rico ou abismalmente pobre.
\newpage
\begin{titulo}
    \raggedright\raggedright\textbf{3. A SUPERAÇÃO DO TRABALHO COMO MANUTENÇÃO DA DESIGUALDADE SOCIAL}
\end{titulo}
\par
A desigualdade social e a desigualdade de gênero no mercado de trabalho são fenômenos complexos que demandam análises aprofundadas. Essas formas de desigualdade têm consequências significativas para os indivíduos, as organizações e a sociedade em geral. A desigualdade social refere-se às disparidades socioeconômicas existentes entre grupos distintos, manifestadas em termos de acesso a oportunidades de emprego, remuneração, benefícios, progressão na carreira e segurança no trabalho. Essa desigualdade é frequentemente resultado de estruturas socioeconômicas e sistemas de poder que perpetuam a concentração de recursos e riqueza nas mãos de poucos, deixando outros grupos em desvantagem.
\par
A  filósofa, escritora e ativista do movimento social negro brasileiro Aparecida Sueli Carneiro destaca em seu artigo \textit{Mulheres em movimento} \cite{carneiro2003mulheres} a significativa disparidade entre negros e brancos no que se refere às posições ocupacionais no país. O movimento de mulheres negras ressalta essa desigualdade, que se torna ainda mais acentuada quando consideramos a interseção de gênero e raça. Apesar dos avanços alcançados pela luta feminista no mercado de trabalho, essas conquistas não foram capazes de eliminar as desigualdades raciais que impedem um maior progresso para as mulheres negras nesse contexto. Dessa forma, as abordagens universalistas da luta das mulheres não apenas revelam sua fragilidade, mas também a impossibilidade de as reivindicações decorrentes dessas abordagens se tornarem viáveis para enfrentar as especificidades do racismo brasileiro.
\par
Com isso, hoje é notório que uma das principais formas de acensão e inclusão social, não só mulheres negras, como relatado por Sueli Carneiro, mas de todos os distintos grupos que são cotidianamente suprimidos socioeconomicamente, é o galgar educacional, conseguir sua mudança de vida e prestigio social a partir da educação e especialização técnica de trabalho. A educação desempenha um papel crucial na capacitação das pessoas, fornecendo conhecimentos, habilidades e competências necessárias para o sucesso em diversas esferas da vida, como o mercado de trabalho e a participação cidadã.
\par
No entanto, quando vislumbramos o crescente domínio das inteligencias artificiais no mercado de trabalho, em suma de nível superior, para uma maior retenção de mais-valia pelo capitalista burguês, tendo em vista que esse tem como foco principal a manutenção do modelo de exploração capitalista, faz-se necessário uma reflexão sobre como todos os grupo minoritários que vêm galgando sua crescente socioeconômica, por meio do trabalho especializado, serão postos novamente no seu primeiro ponto de partida, a desigualdade e exclusão social.
\par
Assim, tendo em vista que o capitalismo é um sistema econômico baseado na propriedade privada dos meios de produção e na busca pelo lucro. Ele se caracteriza pelo mercado e por um falso senso de liberdade de concorrência, onde os bens e serviços são trocados com base na oferta e demanda. No entanto, é intrínseco ao capitalismo a geração e reprodução da desigualdade social. Marx, em \textit{O Capital} \cite{marx2016capital}, argumenta que a propriedade privada dos meios de produção no capitalismo é a base fundamental da desigualdade. A classe capitalista, que detém a propriedade dos recursos produtivos, como fábricas e terras, tem o controle sobre a produção e a distribuição da riqueza. Enquanto isso, a classe trabalhadora, que não possui propriedades e precisa vender sua força de trabalho para sobreviver, fica em desvantagem econômica e social.
\par
Com esse vies, podemos também mencionar Achille Mbembe, filósofo, teórico político e historiador camaronês, no qual em seu livro \textit{Necropolitics} \cite{mbembe2006necropolitics} argumenta que a necropolítica é uma forma de poder que busca controlar e determinar quem pode viver e quem deve morrer. Ele analisa as estratégias utilizadas pelos Estados e outras entidades políticas para exercer controle sobre a vida e a morte das populações, incluindo o uso da violência, da opressão e do genocídio. Mbembe também argumenta que a necropolítica está intrinsecamente ligada ao sistema capitalista, pois a busca por lucro e acumulação de riqueza muitas vezes implica na exploração e desvalorização da vida humana. Ele explora as formas como o capitalismo promove desigualdades sociais e econômicas, bem como a marginalização e o tratamento desumanizador de certos grupos sociais.
\par
Com o exposto, quando pensamos no futuro da inteligencia artificial na superação da mão de obra humana, estamos pensando também na manutenção da desigualdade social e econômica, onde os grandes capitalistas estarão substituindo sua custosa mão de obra qualificada por uma mão de obra inteligente e sem custos salariais. Com isso, teremos mais uma vez uma segregação na sociedade levando a manutenção da necropolítica vigente ao mundo capitalista em que estamos ancorados.
