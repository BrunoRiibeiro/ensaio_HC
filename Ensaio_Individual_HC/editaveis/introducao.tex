\begin{titulo}
    \raggedright\raggedright\textbf{1. INTRODUÇÃO}
\end{titulo}
\par
Há muito vem-se discutindo sobre o futuro das profissões, e a não maior necessidade do homem no mercado de trabalho com o avanço da tecnologia. Assim, como esperado, desde de janeiro de 2023 esta pauta vem crescendo no debate público, devido às recentes notícias e relatos acerca dos avanços da inteligência artificial.
\par
O presente ensaio visa como uma forma de reflexão e análise, por meio de uma perspectiva materialista marxista, apresentar quais as possíveis aspirações e contradições podemos esperar para o futuro do mercado de trabalho e como estes avanços e mudanças podem afetar as relações socioeconômicas não só no Brasil mas também no mundo. Pois, muito se tem dito sobre nos últimos anos, porém sempre com muito pouco embasamento teórico. Assim, por meio de uma abordagem metodológica de revisão de literatura, este ensaio se propõe a tentar responder algumas dessas questões do cerne de nossa atualidade.
\par
O materialismo histórico dialético é uma abordagem teórica central no Manifesto Comunista, escrito por Karl Marx e Friedrich Engels em 1848. Desenvolvido por Marx e Engels, o materialismo histórico dialético constitui uma ferramenta analítica e interpretativa que busca compreender a dinâmica da sociedade e a transformação histórica.
\par
O materialismo histórico dialético parte do pressuposto de que as relações sociais são determinadas pelas condições materiais de produção nas quais os indivíduos de um determinado tempo e espaço histórico existentes em uma determinada sociedade estejam inseridos. Isso implica reconhecer que as formas de organização social, como as estruturas econômicas e políticas, são moldadas pelos modos de produção predominantes em uma determinada época. Por conseguinte, a sociedade é entendida como um sistema complexo em constante desenvolvimento e transformação.
\par
A dialética, no que lhe concerne, é uma abordagem filosófica que avulta a contradição, a mudança e a interação como motores vitais da transformação social. No contexto do materialismo histórico, a dialética é um ferramenta para analisar as contradições inerentes às relações sociais de uma sociedade capitalista. Essas contradições, como a luta de classes entre o proletariado e a burguesia, são miradas como forças para impulsos de mudança social.
\par
Em sua obra \textit{Introdução à contribuição para a crítica da Economia Política}, Marx, de forma mais elaborada, expõe seu método apontando que, tendo em vista a economia política, "Parece mais correto começar pelo que há de mais concreto e real nos dados; assim, pois, na economia, pela população que é a base e sujeito de todo ato social da produção"  \cite{ItroPolit} . Contudo, do pondo de vista do autor, desse modo estaríamos ignorando elementos como o trabalho e o capital, os quais fundamentam a população. Segundo Marx e Engels, a história da humanidade é fundamentada por uma série de lutas e conflitos entre diferentes classes sociais. A evolução da história é entendida como uma sucessão de modos de produção, cada um com suas próprias contradições e potencialidades. Com isso, o materialismo histórico dialético busca analisar as contradições presentes em uma dada sociedade, compreendendo como essas contradições podem levar a transformações sociais radicais.
\par
