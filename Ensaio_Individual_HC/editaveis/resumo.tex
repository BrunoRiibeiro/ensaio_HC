\begin{titulo}
\raggedright\raggedright \textbf{Resumo}:
\end{titulo}
Este texto discute o futuro do mercado de trabalho diante do avanço da tecnologia e da inteligência artificial. Através de uma perspectiva materialista marxista, o texto propõe uma reflexão e análise sobre as possíveis aspirações e contradições que podem surgir nesse contexto, tanto no Brasil quanto no mundo.
O materialismo histórico dialético, abordagem teórica central no Manifesto Comunista de Karl Marx e Friedrich Engels, é utilizado como ferramenta analítica para compreender a dinâmica da sociedade e sua transformação ao longo da história. Segundo essa abordagem, as relações sociais são determinadas pelas condições materiais de produção presentes em uma determinada sociedade, o que implica reconhecer que as estruturas econômicas e políticas são moldadas pelos modos de produção predominantes.
Marx e Engels argumentam no Manifesto Comunista que o capitalismo é uma forma de organização social baseada na exploração e alienação do proletariado. A luta de classes, a concentração de riqueza nas mãos da burguesia e a exploração da força de trabalho são contradições fundamentais que levam à necessidade de uma revolução proletária. A acumulação de riqueza nas mãos da burguesia é possível devido à mais-valia, que é a diferença entre o valor produzido pelo trabalho de um proletário e o valor que ele recebe como salário. Tendo isso em vista, são definições cruciais para a abordagem da temática da substituição da mão de obra humana pela artificial.
A evolução da inteligência artificial (IA) tem sido um dos avanços mais significativos e promissores no campo da tecnologia nas últimas décadas. A inteligência artificial refere-se à capacidade de máquinas e sistemas computacionais imitarem a inteligência humana, aprendendo, raciocinando, reconhecendo padrões e tomando decisões de forma autônoma.
Com os avanços recentes da inteligência artificial, percebe-se que o trabalho a ser substituído não é mais o trabalho físico, mas sim o trabalho intelectual. Essas tecnologias têm a capacidade de realizar tarefas de alto nível de qualificação técnica. No entanto, a preocupação com a qualidade dessas realizações muitas vezes não é relevante para os capitalistas. O objetivo principal é reduzir os custos com mão de obra, aumentando a extração de mais-valia.
Assim, a evolução da IA não necessariamente diminuirá o trabalho da classe trabalhadora ou reduzirá a desigualdade social. Pelo contrário, pode ser usada como uma ferramenta para manter o status quo, com a classe proletária sendo removida de suas posições qualificadas e destinada a trabalhos braçais novamente. Isso pode levar a uma grande ruptura na estrutura de classes sociais, resultando em uma sociedade dividida entre os incrivelmente ricos e os abismalmente pobres.
A desigualdade social e de gênero no mercado de trabalho são fenômenos complexos com consequências significativas. A desigualdade social envolve disparidades socioeconômicas entre grupos, resultantes de estruturas e sistemas de poder que concentram recursos nas mãos de poucos. A necropolítica, vinculada ao capitalismo, controla quem vive e quem morre, promovendo marginalização e desumanização. O futuro da inteligência artificial na substituição da mão de obra humana pode agravar a desigualdade social e econômica, mantendo a segregação e a necropolítica na sociedade capitalista.
 \vspace{\onelineskip}
    
 \textbf{Palavras-chave}: Materialismo. Dialética. Inteligência. Artificial. Empregos.
\pagebreak
